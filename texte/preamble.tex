%
% Pakete
% ========================================
%
%% Schrift festlegen =====================
% Dokumente müssen in UTF-8 Kodierung gespeichert sein
\usepackage[utf8x]{inputenc}

% Setting für Multilang Dokumente mit Deutsch als Hauptsprache
\usepackage[main=ngerman,bidi=default]{babel}
\babelprovide[import=he]{hebrew}
\babelprovide[import=el]{polutonikogreek}
\babelfont[hebrew]{rm}[Contextuals=Alternate]{SBL Hebrew}
\babelfont[polutonikogreek]{rm}[Contextuals=Alternate]{SBL Greek}

% Anführungszeichen steuern: 
% \enquote{text}
% \textquote[〈cite〉][〈punct〉]{〈text〉}〈tpunct〉
\usepackage[babel,german=quotes]{csquotes}

% Verbesserter Textsatz
\usepackage{microtype}

% Grafiken verwenden
\usepackage{graphicx}
% float Umgebungen ausrichten
\usepackage{float}
% zum Ausrichten einer Box
\usepackage{adjustbox}

% schöne Tabellen erstellen
\usepackage{longtable}
% Tabellen zeilen oder Spalten zusammenschließen
\usepackage{multirow}
\usepackage{multicol}
% 3x sub title
\usepackage{caption3}
%Zeilennummern einfügen
% \linenumbers -> nummeriert jede Zeile; \modulolinenumbers[5] nummerriert im Intervall
\usepackage{lineno}

% ToDos mit \todo{} einfügen
% ToDo-Notes erscheinen solange der Status des Dokuments nicht final ist
\usepackage[ngerman,colorinlistoftodos,textsize=footnotesize,obeyFinal]{todonotes}

% Zeilenabstand steuern
%\usepackage[onehalfspacing]{setspace}
\usepackage{setspace}

% \titleref{<label-name>} gibt die Überschrift des Punktes aus
\usepackage{titleref}
% zusätzliche Optionen für Listen
\usepackage{enumitem}

%% Bibliographie Einstellungen ===========================
% zitieren mit \footcite[prenote][postnote]{bib-id}
% ein f. oder ff. nach der Seitennummer erreicht man mit \psq bzw. \psqq
\usepackage[style="../bib-style/MarTheol" %eigener Zitierstil der eigenlich zu verbose-trad3 identisch ist, aber eine Verbesserung enthält. aaO und ebd wird nicht als erstes auf neuen Seiten gesetzt, sonder stattdessen der Kurztitel
,isbn=false
,pagetracker=page
% gleicher Autorenname wird auch dann gesetzt, wenn er mehrfach nacheinander vor kommt
,dashed=false
%,url=false
]{biblatex}
% unnötige BibLatex-Einträge entfernen, dass sie nicht mit im Literaturverzeichnis auftauchen
\AtEveryBibitem{% Clean up the bibtex rather than editing it
	\clearfield{date}
	\clearfield{eprint}
	\clearfield{isbn}
	\clearfield{issn}
	\clearfield{month}
	\clearlist{publisher}
}

% Überschriften für die einzelnen Abschnitte im Literaturverzeichnis
\defbibheading{quell}{\subsection*{Primärliteratur}}
\defbibheading{lit}{\subsection*{Sekundärliteratur}}
\defbibheading{online}{\subsection*{Internetquellen}}
\defbibheading{hilfsm}{\subsection*{Hilfsmittel}}

% erst Nachname, dann Vorname
\DeclareNameAlias{default}{family-given}
\DeclareNameAlias{sortname}{family-given}
% Trenner zwischen den Namen ein Slash
\renewcommand*{\multinamedelim}{\addslash}
\renewcommand*{\finalnamedelim}{\addslash}
\renewcommand*{\multilistdelim}{\addslash}
\renewcommand*{\finallistdelim}{\addslash}
% Doppelpunkt nach Autorenname
\renewcommand*{\labelnamepunct}{\addcolon\space}
% Trennzeichen zwischen einzelen Elementen (z.B. Titel, Band, Verlag)
\renewcommand*{\newunitpunct}{\addcomma\space}
% Im Zitat wird nun der Verlag nicht mehr angezeigt, sondern nur Ort Jahr
\renewbibmacro*{publisher+location+date}{%
	\printlist{location}
	\usebibmacro{date}
	\newunit}

% vollen Autorennamen nennen
\DeclareCiteCommand{\citeauthorfirstlast}
{\boolfalse{citetracker}%
	\boolfalse{pagetracker}%
	\DeclareNameAlias{labelname}{first-last}%
	\usebibmacro{prenote}}
{\ifciteindex
	{\indexnames{labelname}}
	{}%
	\printnames{labelname}}
{\multicitedelim}
{\usebibmacro{postnote}}


%% als letzte Pakete laden!
%\usepackage[pdfpagelabels=true,german,backref,pagebackref]{hyperref}
%\usepackage[pdfpagelabels=true,german,pagebackref]{hyperref}
%\hypersetup{
%  pdftitle    		= {\pdfTitle}
%  ,pdfsubject  		= {\pdfSubject}
%  ,pdfauthor   		= {\pdfAutor}
%  ,pdfkeywords 		= {\pdfKeywords}
%	,bookmarksopen 	= {true}
%	,colorlinks			=	true
%	,breaklinks			=	true
%	,linkcolor			=	black
%	,urlcolor				=	blue
%	%,pagecolor				=	black
%	,urlcolor				=	blue
%	,citecolor			=	blue
%	,bookmarksopen	=	true
%	%,bookmarksopenlevel=0
%	,bookmarksnumbered=true
%	,plainpages			=	false
%	%,pdftoolbar			=	true
%	%,pdfstartview		=% leer lassen, damit der Reader seine Standardeinstellung verwendet
%	%,pdfpagemode		=	FullScreen
%	%,pdffitwindow		=	true
%	,pdfborder			=	{0 0 0}
%}

%\usepackage[a4paper
%	,left=22mm
%	,right=50mm
%	,top=22mm
%	,bottom=27mm
%%	,marginparsep=2pt
%	]{geometry}
\usepackage[a4paper,margin=25mm]{geometry}
	
% Ende
% ========================================
%%
% Einstellungen
% ========================================
% Randbreite für ToDo Notes festlegen
\setlength{\marginparwidth}{36mm}


% mit \blankpage kann nun eine leere Seite eingefügt werden
\def\blankpage{%
	\clearpage%
	\thispagestyle{empty}%
	% dieser Befehl steuert, ob die Seitenzahl weiterläuft (ja, wenn auskommentiert)
%	\addtocounter{page}{-1}%
	\null%
	\clearpage}
