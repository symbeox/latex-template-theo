%% Dokumentenklasse (Koma Script) -----------------------------------------
\documentclass[
   draft,     % Entwurfsstadium
	% final am Ende einfügen, damit todonotes verschwinden
   %final,      % fertiges Dokument
%%%% --- Schriftgröße ---
   11pt,
   %smallheadings,		% kleine Überschriften
   %normalheadings,		% normale Überschriften
   %bigheadings,			% große Überschriften
%%%% --- Sprache ---
   ngerman,           % wird an andere Pakete weitergereicht
%%%% === Seitengröße ===
   % letterpaper,
   % legalpaper,
   % executivepaper,
   a4paper,
   % a5paper,
   % landscap,
%%%% === Optionen für den Satzspiegel ===
   %DIVcalc,         % automatische Berechnung einer guten Zeilenlaenge
   1.1headlines,     % Zeilenanzahl der Kopfzeilen
   %headinclude,     % Kopf einbeziehen
   %headexclude,      % Kopf nicht einbeziehen
   %footinclude,     % Fuss einbeziehen
   %footexclude,      % Fuss nicht einbeziehen
   %mpinclude,       % Margin einbeziehen
   %mpexclude,        % Margin nicht einbeziehen
   pagesize,         % Schreibt die Papiergroesse in die Datei.
                     % Wichtig fuer Konvertierungen
%%%% === Layout ===
   oneside,         % einseitiges Layout
   %twoside,          % Seitenraender für zweiseitiges Layout
   %onecolumn,        % Einspaltig
   %twocolumn,       % Zweispaltig
   %openany,         % Kapitel beginnen auf jeder Seite
   %openright,        % Kapitel beginnen immer auf der rechten Seite
                     % (macht nur bei 'twoside' Sinn)
   %cleardoubleplain,    % leere, linke Seite mit Seitenstil 'plain'
   %cleardoubleempty,    % leere, linke Seite mit Seitenstil 'empty'
   titlepage,        % Titel als einzelne Seite ('titlepage' Umgebung)
   %notitlepage,     % Titel in Seite integriert
%%%% --- Absatzeinzug ---
   %                 % Absatzabstand: Einzeilig,
   %parskip,         % Freiraum in letzter Zeile: 1em
   %parskip*,        % Freiraum in letzter Zeile: Viertel einer Zeile
   %parskip+,        % Freiraum in letzter Zeile: Drittel einer Zeile
   %parskip-,        % Freiraum in letzter Zeile: keine Vorkehrungen
   %                 % Absatzabstand: Halbzeilig
   %halfparskip,     % Freiraum in letzter Zeile: 1em
   %halfparskip*,    % Freiraum in letzter Zeile: Viertel einer Zeile
   %halfparskip+,    % Freiraum in letzter Zeile: Drittel einer Zeile
   %halfparskip,     % Freiraum in letzter Zeile: keine Vorkehrungen
   %                 % Absatzabstand: keiner
   %parindent,        % Eingerückt (Standard)
%%%% --- Kolumnentitel ---
   headsepline,      % Linie unter Kolumnentitel
   %headnosepline,   % keine Linie unter Kolumnentitel
   %footsepline,     % Linie unter Fussnote
   %footnosepline,   % keine Linie unter Fussnote
%%%% --- Kapitel ---
   %chapterprefix,   % Ausgabe von 'Kapitel:'
   nochapterprefix,  % keine Ausgabe von 'Kapitel:'
%%%% === Verzeichnisse (TOC, LOF, LOT, BIB) ===
   %liststotoc,      % Tabellen & Abbildungsverzeichnis ins TOC
   %idxtotoc,        % Index ins TOC
   %bibtotoc,         % Bibliographie ins TOC
   %bibtotocnumbered, % Bibliographie im TOC nummeriert
   %liststotocnumbered, % Alle Verzeichnisse im TOC nummeriert
   %tocindent,        % eingereuckte Gliederung
   %tocleft,         % Tabellenartige TOC
   %listsindent,      % eingereuckte LOT, LOF
   %listsleft,       % Tabellenartige LOT, LOF
   %pointednumbers,  % Überschriftnummerierung mit Punkt, siehe DUDEN !
   %pointlessnumbers, % Überschriftnummerierung ohne Punkt, siehe DUDEN !
   %openbib,         % alternative Formatierung des Literaturverzeichnisses
%%%% === Matheformeln ===
   %leqno,           % Formelnummern links
   %fleqn,            % Formeln werden linksbuendig angezeigt
]{scrartcl}%     Klassen: scrartcl, scrreprt, scrbook
% -------------------------------------------------------------------------

%%% Preambel
%
% Pakete
% ========================================
%
%% Schrift festlegen =====================
% Dokumente müssen in UTF-8 Kodierung gespeichert sein
\usepackage[utf8x]{inputenc}

% Setting für Multilang Dokumente mit Deutsch als Hauptsprache
\usepackage[main=ngerman,bidi=default]{babel}
\babelprovide[import=he]{hebrew}
\babelprovide[import=el]{polutonikogreek}
\babelfont[hebrew]{rm}[Contextuals=Alternate]{SBL Hebrew}
\babelfont[polutonikogreek]{rm}[Contextuals=Alternate]{SBL Greek}

% Anführungszeichen steuern: 
% \enquote{text}
% \textquote[〈cite〉][〈punct〉]{〈text〉}〈tpunct〉
\usepackage[babel,german=quotes]{csquotes}

% Verbesserter Textsatz
\usepackage{microtype}

% Grafiken verwenden
\usepackage{graphicx}
% float Umgebungen ausrichten
\usepackage{float}
% zum Ausrichten einer Box
\usepackage{adjustbox}

% schöne Tabellen erstellen
\usepackage{longtable}
% Tabellen zeilen oder Spalten zusammenschließen
\usepackage{multirow}
\usepackage{multicol}
% 3x sub title
\usepackage{caption3}
%Zeilennummern einfügen
% \linenumbers -> nummeriert jede Zeile; \modulolinenumbers[5] nummerriert im Intervall
\usepackage{lineno}

% ToDos mit \todo{} einfügen
% ToDo-Notes erscheinen solange der Status des Dokuments nicht final ist
\usepackage[ngerman,colorinlistoftodos,textsize=footnotesize,obeyFinal]{todonotes}

% Zeilenabstand steuern
%\usepackage[onehalfspacing]{setspace}
\usepackage{setspace}

% \titleref{<label-name>} gibt die Überschrift des Punktes aus
\usepackage{titleref}
% zusätzliche Optionen für Listen
\usepackage{enumitem}

%% Bibliographie Einstellungen ===========================
% zitieren mit \footcite[prenote][postnote]{bib-id}
% ein f. oder ff. nach der Seitennummer erreicht man mit \psq bzw. \psqq
\usepackage[style="../bib-style/MarTheol" %eigener Zitierstil der eigenlich zu verbose-trad3 identisch ist, aber eine Verbesserung enthält. aaO und ebd wird nicht als erstes auf neuen Seiten gesetzt, sonder stattdessen der Kurztitel
,isbn=false
,pagetracker=page
% gleicher Autorenname wird auch dann gesetzt, wenn er mehrfach nacheinander vor kommt
,dashed=false
%,url=false
]{biblatex}
% unnötige BibLatex-Einträge entfernen, dass sie nicht mit im Literaturverzeichnis auftauchen
\AtEveryBibitem{% Clean up the bibtex rather than editing it
	\clearfield{date}
	\clearfield{eprint}
	\clearfield{isbn}
	\clearfield{issn}
	\clearfield{month}
	\clearlist{publisher}
}

% Überschriften für die einzelnen Abschnitte im Literaturverzeichnis
\defbibheading{quell}{\subsection*{Primärliteratur}}
\defbibheading{lit}{\subsection*{Sekundärliteratur}}
\defbibheading{online}{\subsection*{Internetquellen}}
\defbibheading{hilfsm}{\subsection*{Hilfsmittel}}

% erst Nachname, dann Vorname
\DeclareNameAlias{default}{family-given}
\DeclareNameAlias{sortname}{family-given}
% Trenner zwischen den Namen ein Slash
\renewcommand*{\multinamedelim}{\addslash}
\renewcommand*{\finalnamedelim}{\addslash}
\renewcommand*{\multilistdelim}{\addslash}
\renewcommand*{\finallistdelim}{\addslash}
% Doppelpunkt nach Autorenname
\renewcommand*{\labelnamepunct}{\addcolon\space}
% Trennzeichen zwischen einzelen Elementen (z.B. Titel, Band, Verlag)
\renewcommand*{\newunitpunct}{\addcomma\space}
% Im Zitat wird nun der Verlag nicht mehr angezeigt, sondern nur Ort Jahr
\renewbibmacro*{publisher+location+date}{%
	\printlist{location}
	\usebibmacro{date}
	\newunit}

% vollen Autorennamen nennen
\DeclareCiteCommand{\citeauthorfirstlast}
{\boolfalse{citetracker}%
	\boolfalse{pagetracker}%
	\DeclareNameAlias{labelname}{first-last}%
	\usebibmacro{prenote}}
{\ifciteindex
	{\indexnames{labelname}}
	{}%
	\printnames{labelname}}
{\multicitedelim}
{\usebibmacro{postnote}}


%% als letzte Pakete laden!
%\usepackage[pdfpagelabels=true,german,backref,pagebackref]{hyperref}
%\usepackage[pdfpagelabels=true,german,pagebackref]{hyperref}
%\hypersetup{
%  pdftitle    		= {\pdfTitle}
%  ,pdfsubject  		= {\pdfSubject}
%  ,pdfauthor   		= {\pdfAutor}
%  ,pdfkeywords 		= {\pdfKeywords}
%	,bookmarksopen 	= {true}
%	,colorlinks			=	true
%	,breaklinks			=	true
%	,linkcolor			=	black
%	,urlcolor				=	blue
%	%,pagecolor				=	black
%	,urlcolor				=	blue
%	,citecolor			=	blue
%	,bookmarksopen	=	true
%	%,bookmarksopenlevel=0
%	,bookmarksnumbered=true
%	,plainpages			=	false
%	%,pdftoolbar			=	true
%	%,pdfstartview		=% leer lassen, damit der Reader seine Standardeinstellung verwendet
%	%,pdfpagemode		=	FullScreen
%	%,pdffitwindow		=	true
%	,pdfborder			=	{0 0 0}
%}

%\usepackage[a4paper
%	,left=22mm
%	,right=50mm
%	,top=22mm
%	,bottom=27mm
%%	,marginparsep=2pt
%	]{geometry}
\usepackage[a4paper,margin=25mm]{geometry}
	
% Ende
% ========================================
%%
% Einstellungen
% ========================================
% Randbreite für ToDo Notes festlegen
\setlength{\marginparwidth}{36mm}


% mit \blankpage kann nun eine leere Seite eingefügt werden
\def\blankpage{%
	\clearpage%
	\thispagestyle{empty}%
	% dieser Befehl steuert, ob die Seitenzahl weiterläuft (ja, wenn auskommentiert)
%	\addtocounter{page}{-1}%
	\null%
	\clearpage}

%%% Bibliographie-Datenbank
\bibliography{test_bib.bib}

%% Dokument Beginn %%%%%%%%%%%%%%%%%%%%%%%%%%%%%%%%%%%%%%%%%%%%%%%%%%%%%%%%
\begin{document}
	% --- Deckblatt
	\begin{titlepage}
	\newgeometry{margin=25mm}
	\begin{large}
		\noindent{Universität\\
		Theologische Fakultät\\
		SE: <SE-Titel>\\
		Leitung: <Leitung>\\
		<Art der Arbeit (Proseminararbeit)>\\
		Sommersemester 2022\\
		Datum: \today\par}
	\end{large}
	
	%% Je nach Titel muss man mit diesen Abständen etwas probieren, bist es gut aussieht
	\vspace{5cm}
	\begin{center}
		{\huge\bfseries Das Thema meiner wissenschaftlichen Hausarbeit\par}
		{\LARGE\bfseries evtl. Untertitel\par}
		\vspace{1cm}
	\end{center}
	
	%% Je nach Titel muss man mit diesen Abständen etwas probieren, bist es gut aussieht
	\vspace{7cm}
	\begin{large}
		\noindent{Verfasser: Vorname Name\\
		Matrikel-Nr.: 12345678\\
		Fachsemester: 1\\
		Mobil: 012345678\\
		Email: Mustermann@uni.de
		\par}
		\vspace{-3mm}
		\begin{longtable}{@{}p{5cm}|p{\textwidth}}
			Semesteradresse & Heimatadresse \\
			Musterweg 3 & Musterweg 3\\
			12345 Musterhausen & 12345 Musterhausen\\
		\end{longtable}
	\end{large}
	\vfill
\end{titlepage}
\restoregeometry
	
	% --- Inhaltsverzeichnis
	\newgeometry{margin=25mm}
	\tableofcontents
	\clearpage
	
	% --- Hauptteil
	% 1.5 Zeilenabstand für den Hauptteil einschalten
	\onehalfspacing
	% Korrekturrand
	\newgeometry{margin=25mm,right=50mm}
	% Hauptdatei mit Gliederung und allen Inhalten
	%% Eizelne Kapitel und Abschnitte importieren

\section{Voraussetzungen}
\begin{itemize}
	\item BibLatex mit Biber als Backend
	\item xelatex oder lualatex als Kompiler, damit Hebräisch verwendet werden kann
	\item SBL Schriftarten für Griechisch und Hebräisch\\ (Download: https://www.sbl-site.org/educational/biblicalfonts.aspx)
\end{itemize}
\section{ToDo Notes}
Mit dem todonotes-Paket lassen sich einfach ToDos erstellen, die am Ende des Dokuments auch in einer Liste zusammengefasst werden.

An der gewünschten Stelle im Text verwendet man einfach \textbackslash todo\{<Anmerkung oder Aufgabe>\} und erhält damit eine Anmerkung am Seitenrand\todo{so wie hier}.

\textbf{Wichtig!} Solange der Status des gesamten Dokuments in der Master.tex Datei auf \textit{draft} steht (Zeile 3), werden die ToDos angezeigt. Für die Abgabe des Dokuments sollte \textit{draft} in \textit{final} geändert werden, damit verschwinden auch die ToDos und die zusammenfassende Liste am Ende des Dokuments.
\section{Zitation}
\subsection{csquotes für An- und Ausführungszeichen}
Um Zitate in An- und Ausführungszeichen zu hüllen verwendet diese Vorlage den \textbackslash enquote\{\}-Befehl. Das hat den Vorteil, dass sich die An- und Ausführungszeichen per Einstellung verschiedenen Sprachen anpassen und auch verschachtelte Zitate mit doppelten und einfachen An- und Ausführungszeichen automatisch richtig dargestellt werden.

Beispiel: \enquote{Hier ist ein lange Abschnitt der zitiert wird. Dieser Abschnitt enthält ebenfalls ein \enquote{Zitat aus einem zweiten Werk} wofür einfache An- und Ausführungszeichen benötigt werden.}

\subsection{shorttitle}
Das Attribut \textit{shorttitle} ist für den gewählten Zitierstil sehr wichtig. Beim ersten Vorkommen eines Werks wird dieses in der Fußnote vollständig zitiert, aber bei jedem weiteren mit einer Kurzform. Dafür muss ein Kurztitel angegeben werden, der meist aus einem oder zwei aussagekräftigen Worten aus dem eigentlichen Titel des Werks besteht.

\subsection{Zitate einfügen}
\enquote{Lorem ipsum dolor sit amet, consetetur sadipscing elitr, sed diam nonumy eirmod tempor invidunt ut labore et dolore magna aliquyam erat, sed diam voluptua.}\footcite[104]{Gnilka.1978} At vero eos et accusam et justo duo dolores et ea rebum. Stet clita kasd gubergren, no sea takimata sanctus est Lorem ipsum dolor sit amet. Lorem ipsum dolor sit amet, consetetur sadipscing elitr, sed diam nonumy eirmod tempor invidunt ut labore et dolore magna aliquyam erat, sed diam voluptua. At vero eos et accusam et justo duo dolores et ea rebum. Stet clita kasd gubergren, no sea takimata sanctus est Lorem ipsum dolor sit amet. Lorem ipsum dolor sit amet, consetetur sadipscing elitr, sed diam nonumy eirmod tempor invidunt ut labore et dolore magna aliquyam erat, sed diam voluptua. At vero eos et accusam et justo duo dolores et ea rebum. Stet clita kasd gubergren, no sea takimata sanctus est Lorem ipsum dolor sit amet.\footcite[Vgl.][104]{Gnilka.1978}

\enquote{Duis autem vel eum iriure dolor in hendrerit in vulputate velit esse molestie consequat, vel illum dolore eu feugiat nulla facilisis at vero eros et accumsan et iusto odio dignissim qui blandit praesent luptatum zzril delenit augue duis dolore te feugait nulla facilisi. Lorem ipsum dolor sit amet, consectetuer adipiscing elit, sed diam nonummy nibh euismod tincidunt ut laoreet dolore magna aliquam erat volutpat.}\footcite[207]{Gnilka.1978}

Ut wisi enim ad minim veniam, quis nostrud exerci tation ullamcorper suscipit lobortis nisl ut aliquip ex ea commodo consequat. Duis autem vel eum iriure dolor in hendrerit in vulputate velit esse molestie consequat, vel illum dolore eu feugiat nulla facilisis at vero eros et accumsan et iusto odio dignissim qui blandit praesent luptatum zzril delenit augue duis dolore te feugait nulla facilisi.\footcites[Vgl.][]{Ego.2007}[Vgl.][289]{Gnilka.1978}

\subsection{Bibliografie}
Die Bibliogafie lässt sich insgesamt in vier Teile untergliedern: Primär-, Sekundärliteratur, Internetquellen und Hilfsmittel. Die Onlinequellen werden über ihren Typ @online herausgefiltert. Primärliteratur muss als einziges keyword \enquote*{quell} haben und Hilfsmittel das keyword \enquote*{hilfsm}. Die restliche Literatur wird als Sekundärliteratur eingeordnet.
\section{Demonstration der Mehrsprachigkeit}
\textsuperscript{1}Im Anfang war das Wort, und das Wort war bei Gott, und Gott war das Wort. \textsuperscript{2}Dasselbe war im Anfang bei Gott.
(Joh 1,1–2)

\selectlanguage{polutonikogreek}

\textsuperscript{1} Ἐν ἀρχῇ ἦν ὁ λόγος, καὶ ὁ λόγος ἦν πρὸς τὸν θεόν, καὶ θεὸς ἦν ὁ λόγος. \textsuperscript{2}Οὗτος ἦν ἐν ἀρχῇ πρὸς τὸν θεόν.
\foreignlanguage{british}{(John 1:1–2)}

\selectlanguage{hebrew}

\textsuperscript{1}בְּרֵאשִׁ֖ית בָּרָ֣א אֱלֹהִ֑ים אֵ֥ת הַשָּׁמַ֖יִם וְאֵ֥ת הָאָֽרֶץ׃
\textsuperscript{2}וְהָאָ֗רֶץ הָיְתָ֥ה תֹ֨הוּ֙ וָבֹ֔הוּ וְחֹ֖שֶׁךְ עַל־פְּנֵ֣י תְהֹ֑ום וְר֣וּחַ אֱלֹהִ֔ים מְרַחֶ֖פֶת
עַל־פְּנֵ֥י הַמָּֽיִם׃ \foreignlanguage{british}{(Genesis 1:1–2)}

\selectlanguage{ngerman}

Inline Greek (\foreignlanguage{polutonikogreek}{Ἐν ἀρχῇ ἦν ὁ λόγος} [John 1:1]) and Hebrew (\foreignlanguage{hebrew}{בְּרֵאשִׁ֖ית בָּרָ֣א אֱלֹהִ֑ים אֵ֥ת הַשָּׁמַ֖יִם וְאֵ֥ת הָאָֽרֶץ} [Genesis
1:1]) also must work.

\clearpage
\section*{Eigenständigkeitserklärung}
Ich versichere hiermit, dass ich die vorliegende Arbeit selbständig verfasst, ganz oder in Teilen noch nicht als Prüfungsleistung vorgelegt und keine anderen als die angegebenen Hilfsmittel und Literatur benutzt habe. Sämtliche Stellen der Arbeit, die benutzten Werken im Wortlaut oder dem Sinn nach entnommen sind, habe ich durch Quellenangaben kenntlich gemacht. Dies gilt auch für Zeichnungen, Skizzen, bildliche Darstellungen und dergleichen sowie für Quellen aus dem Internet. Bei Zuwiderhandlung wird die Studienleistung aus der Lehrveranstaltung (Seminar zur Einführung, Seminar) bzw. die Prüfungsleistung des Moduls oder die Sprachprüfung mit \enquote{nicht ausreichend} bewertet, der Leistungsnachweis bzw. die Modulprüfung ist nicht bestanden. Ich bin darüber aufgeklärt worden, dass es sich bei Plagiarismus und Täuschungsversuchen um schweres akademisches Fehlverhalten handelt.


\vspace{1.2cm}
\noindent{}Ort, \the\day.\the\month.\the\year\hfill{}Unterschrift: \rule{4.3cm}{.4pt}\par
\clearpage
	%normale Seitenränder wiederherstellen
	\restoregeometry
	% 1.5 Zeilenabstand nach dem Hauptteil ausschalten
	\singlespacing
	\clearpage
	
	% --- Literatur- und andere Verzeichnisse
	\section*{Bibliografie}
	% Bei Quellen ein Keyword einfügen!
	% alle Dinge die als Keyword 'quell' haben werden hier gedruckt
	\printbibliography[heading=quell,keyword=quell]
	% alle restlichen Dinge landen hier
	\printbibliography[heading=lit,nottype=online,notkeyword=quell,notkeyword=hilfsm]
	% alle Dinge vom Type Onlinequelle (@online) landen hier
	\printbibliography[heading=online,type=online]
	% alle Dinge die als Keyword 'hilfsm' haben werden hier gedruckt
	\printbibliography[heading=hilfsm,keyword=hilfsm]
	
	\clearpage
	\listoftodos
%	\listoffigures
%	\listoftables
	
	% --- Anhang
	%\appendix
	
	%% Dokument ENDE %%%%%%%%%%%%%%%%%%%%%%%%%%%%%%%%%%%%%%%%%%%%%%%%%%%%%%%%%%
\end{document}